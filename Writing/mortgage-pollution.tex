\documentclass[a4paper,11pt]{article}
\setlength{\parskip}{\baselineskip}
%\usepackage[latin1]{inputenc}
%\usepackage[english]{babel}
\usepackage[semicolon, authoryear]{natbib}
\usepackage{makecell}
\usepackage{amsmath,amsxtra,amssymb,latexsym,amscd,amsthm}
\usepackage{multicol,multirow}
\usepackage{caption}
\usepackage{color,picture}
\usepackage{graphicx}
%\usepackage{apalike}
%\usepackage{lineno}
%\usepackage{harvard}
%\usepackage[dvipdfm]{graphicx}
\usepackage{natbib}
\usepackage{float,url}
\usepackage{titlesec}
\usepackage{rotating,array}
\usepackage{pst-eps}
\usepackage{longtable}
\usepackage{pst-node}
\usepackage{tikz}
\usepackage{booktabs}
\usepackage{siunitx}
\usepackage{fullpage}
\usepackage{setspace}
\usepackage{lscape}
\setcitestyle{authoryear,open={(},close={)},citesep={;}}

\usepackage[hyperfootnotes=true,pdfpagemode=UseThumbs,pdfstartview=FitH,colorlinks,frenchlinks,breaklinks=true,linkcolor=blue, citecolor=blue]{hyperref}
\usepackage[paper=a4paper, top=2.5cm, bottom=2.5cm, left=3cm, right=2.5cm] {geometry}
\usepackage{hyperref}
%\doublespacing
\linespread{1.3}

\setlength{\parindent}{0pt} %khong thut dau dong 
%\setlength{\parindent}{2em} 

%\bibliographystyle{abbrvnat}
%\setcitestyle{authoryear,open={(},close={)},citesep={;}}

\begin{document}
	
	\title{\textbf{Another cost of living in pollution: The cost of mortgage credit}} 
	
\author{Kien Hoang-Le$^{a, *}$  \medskip \\
	$^{a}$\footnotesize \emph{University of Nottingham (United Kingdom)} \\
	$^{*}$\footnotesize \emph{Corresponding author. E-mail: \texttt{xxx@xxx.com}}
	\\
}	
	
\date{}
\maketitle
%\vspace{0.2cm}
	
	
	
	\begin{abstract}
		
		XXX		
		
		\medskip
		
		\noindent \emph{Keywords}: xxx; xxx; ... \\ 
		\smallskip
		
		\noindent \emph{JEL Classifications}: xxx; xxx; xxx
		
	\end{abstract}

	
	
	\newpage

\newpage

\textbf{Word count}: xxx. \newline

\textbf{Funding details}: This research did not receive any specific grant from funding agencies in the public, commercial, or
not-for-profit sectors. \newline

%\textbf{Statement of interest}: The authors declare no conflict of interest. \newline

\newpage

%\linenumbers

\section{Introduction}\label{intro}

Technology adoption plays a pivotal role in driving development and progress, both at the macroeconomic level and within individual firms. In today's rapidly evolving global landscape, characterized by the Fourth Industrial (I.4.0), the integration of advanced technologies has become increasingly vital for achieving sustainable growth and competitiveness. Technology adoption serves as a catalyst for development by enabling nations and firms to harness the benefits of innovation and enhance their productivity and efficiency \citep{united2018world}. At a macroeconomic level, countries that successfully embrace and integrate new technologies experience improved economic performance, increased labour productivity, and enhanced standards of living. Technological advancements fuel innovation, create new industries, generate employment opportunities, and foster knowledge spillovers, leading to long-term economic growth and development. Within firms, technology adoption is instrumental in unlocking a multitude of benefits. By embracing advanced technologies, firms can optimize their operations, streamline processes, and enhance productivity. Efficient technology adoption enables firms to respond to market demands swiftly, stay ahead of competitors, and capitalism on emerging opportunities. Furthermore, technology adoption can revolutionize products and services, enabling firms to deliver enhanced customer experiences, create new markets, and drive revenue growth.

However, the adoption of advanced technologies is not uniform across the world. Due to capacity disparities, there is a huge lag in technology adoption in developing countries compared to that in developed countries. This is the results of the prolonged challenges developing or least developed countries have continued to face for many years that lead to infrastructure limitations, financial constraints, or skills and education gaps. Moreover, as stated in one recent World Bank report, \cite{cirera2022bridging} pointed out that the technological divide occurs not only across countries but also within them. The disproportionate adopting speed among firms in not only developing countries but also developed countries could exacerbate problems with income inequality across and within nations. Given that more capable and technologically advanced enterprises are also more robust, the technical gap between firms also impacts how differently they are able to handle and recover from economic shocks \citep{cirera2022bridging}.

The remaining of the paper would be organised as follow. Section \ref{lit} will provide an overview on the literature about firms' technology adoption and investment decisions. It will also review previous studies on firms' resources as a foundation or barriers for them to adopt new technologies. Section \ref{bgdata} will lay out the background of the study, the datasets, and the empirical situation of I.4.0 technology adoption in Vietnam. We will present the key measurements in this section as well. After that, section \ref{method} will briefly introduce the techniques used to answer the research questions. Section \ref{result} will report and discuss the estimation results and section \ref{conclu} concludes the paper.

\section{Literature review}\label{lit}

\section{Background, data, and key measurements}\label{bgdata}

\subsection{Data}\label{data}

\subsection{Background}\label{bg}

\subsection{Key measurements}\label{measure}

\newpage

\begin{landscape}
	\begin{table}[htbp]\centering
\def\sym#1{\ifmmode^{#1}\else\(^{#1}\)\fi}
\caption{Example tables \label{tab-exp}}
\begin{tabular}{l*{4}{c}}
\hline\hline
                    &\multicolumn{1}{c}{(1)}         &\multicolumn{1}{c}{(2)}         &\multicolumn{1}{c}{(3)}         &\multicolumn{1}{c}{(4)}         \\
                    &        reg1         &        reg2         &        reg3         &        reg4         \\
\hline
Headroom (in.)      &    -753.778         &                     &       0.047         &                     \\
                    &   (255.512)         &                     &     (0.691)         &                     \\
Trunk space (cu. ft.)&      58.919         &                     &      -0.184         &                     \\
                    &    (78.718)         &                     &     (0.157)         &                     \\
Weight (lbs.)       &                     &       5.041\sym{*}  &                     &      -0.004\sym{*}  \\
                    &                     &     (0.485)         &                     &     (0.002)         \\
Length (in.)        &                     &     -72.947         &                     &      -0.074         \\
                    &                     &    (49.735)         &                     &     (0.058)         \\
Turn circle (ft.)   &     -39.093\sym{*}  &    -145.993         &      -0.626\sym{***}&      -0.176         \\
                    &     (5.727)         &   (119.909)         &     (0.190)         &     (0.194)         \\
Displacement (cu. in.)&      29.537\sym{**} &       8.902         &      -0.005         &       0.008         \\
                    &     (1.276)         &     (5.020)         &     (0.011)         &     (0.011)         \\
Gear ratio          &    -438.598         &      22.532         &       5.201\sym{**} &       2.446         \\
                    &   (278.602)         &   (576.681)         &     (2.043)         &     (1.749)         \\
Car origin          &    3324.469\sym{**} &    3540.934\sym{**} &      -5.094\sym{***}&      -2.776\sym{**} \\
                    &   (146.881)         &   (115.530)         &     (1.649)         &     (1.285)         \\
\hline
Repair FE           &         Yes         &          No         &         Yes         &          No         \\
Clustered Std Err   &         Yes         &         Yes         &          No         &          No         \\
No. of Obs.         &          69         &          74         &          69         &          74         \\
Outcome mean        &    6146.043         &    6165.257         &      21.290         &      21.297         \\
R2                  &       0.524         &       0.573         &       0.694         &       0.686         \\
\hline\hline
\multicolumn{5}{l}{\footnotesize Standard errors in parentheses.}\\
\multicolumn{5}{l}{\footnotesize * p $<$ 0.10, ** p $<$ 0.05, *** p $<$ 0.01.}\\
\end{tabular}
\end{table}

\end{landscape}

\newpage

%\begin{figure}[h]
%	\centering
%	\includegraphics[width=0.8\textwidth]{fig-concept.png}
%	\caption{Adapted conceptual framework}
%	\label{fig-sla}
%\end{figure}

\section{Estimation method}\label{method}

The ordered logistic regression is specified as:
\begin{multline}
	Maximum \ adoption_i^* = \beta_{1,2} Financial_i + \beta_{3,4} HR_i + \beta_{5,6,7} Practicality_i \\
	+ \beta_k X_i + \delta_i + \gamma_i + \lambda_i + \epsilon_i
\end{multline}
where
\begin{align}	
	Maximum \ adoption_i = \left\{
	\begin{matrix}
		&1 &\mbox{ if } &      &  &Maximum \ adoption_i^* &\le &\mu_1 \\
		&2 &\mbox{ if } &\mu_1 &< &Maximum \ adoption_i^* &\le &\mu_2 \\
		&3 &\mbox{ if } &\mu_2 &< &Maximum \ adoption_i^* &\le &\mu_3 \\
		&4 &\mbox{ if } &\mu_3 &< &Maximum \ adoption_i^* &\le &\mu_4 \\
		&5 &\mbox{ if } &\mu_4 &< &Maximum \ adoption_i^* &    &
	\end{matrix}
	\right.
\end{align}
with $\mu_j, (j = \overline{1,4})$ are 4 latent thresholds corresponding to 5 levels of $Maximum \ adoption$. $Financial$, $HR$, and $Practicality$ are three groups of variable of interest presented in the subsection \nameref{measure}. $\delta_i$, $\gamma_i$, and $\lambda_i$ are type, province, and sector fixed effect, respectively. These variables control for firms' types of ownership, province of location, and sector of operation. $X_i$ is a $k \times i$ matrix of control variables, and $\epsilon_i$ is the robust error term.

The binary logistics regression is specified as:
\begin{multline}
	log\left( \frac{P[Adopted \ in \ use = 1|Z=z]}{1- P[Adopted \ in \ use = 1|Z=z]} \right) = \\ 
	\beta_0 + \beta_{1,2} Financial_i + \beta_{3,4} HR_i + \beta_{5,6,7} Practicality_i \\
	+ \beta_k X_i + \delta_i + \gamma_i + \lambda_i + \epsilon_i
\end{multline}
with $P(.)$ is the probability of the variable $Adopted \ in \ use$ equal 1 on the condition of all independent variables included, denoted by $Z$.

For the $No. \ of \ techs \ adopted$ variable, we use zero-inflated Poisson regression with the same model specification. We use the three perceived barriers firms have given including ``financial barrier'', ''human resource barrier'', and ''practicality barrier'' as the predictors for the \textit{zero} observations in the sample.

The binary IPW-RA model can be simply written as:
\begin{equation}
	Difference = \frac{1}{N_T}\sum_{i=1}^{N_T} \left( E[Barrier_1 | Adopted = 1] - E[Barrier_0 | Adopted = 1] \right)
\end{equation}
where $Barrier_1$ and $Barrier_0$ are the potential outcomes if the unit belongs to the treatment or control group; $Adopted = 1$ denotes when a firm adopted a technology; $N_T$ is the number of adopted firms. Yet, since $E[Barrier_0 | Adopted = 1]$ is unobservable, we must calculate the probability of adoption via propensity score. It can be done simply by estimating a probit or logit model with $Adopted \ in \ use$ as a dependent variable and firms' characteristics as independent variables. The propensity score $P(Adopted|Z_i)$, with $Z_i$ is the matrix of observable covariates, is then used as the weight for the outcome models.
\begin{equation}
	\mbox{Inverse propability} = \left\{ \begin{array}{cl}
		\frac{1}{P(Adopted|Z_i)} & \mbox{if } \ Adopted = 1 \\
		\frac{1}{1-P(Adopted|Z_i)} & \mbox{if } \ Adopted = 0
	\end{array} \right.
	.
\end{equation}
Given that, we can calculate the difference between firms adopting technologies and firms not as:
\begin{equation}
	Difference = \frac{1}{N_T}\sum_{i=1}^{N_T}  \left(  \left[  \frac{\widehat{Barriers}_{Adopted=1}}{\hat P(Adopted|Z_i)}\right] - \left[  \frac{\widehat{Barrier}_{Adopted=0}}{1- \hat P(Adopted|Z_i)}\right]\right)  
\end{equation}
with
\begin{equation}
	\left\{ \begin{array}{cl}
		\widehat{Barrier}_{Adopted=1} = \left( \hat{\beta}_{0} + \hat{\beta}_{} Z_{i} \right)_{Adopted=1} \\
		\widehat{Barrier}_{Adopted=0} = \left( \hat{\beta}_{0} + \hat{\beta}_{} Z_{i} \right)_{Adopted=0}
	\end{array} \right.
\end{equation}
are the adjusted regression outcome models. 

\section{Results and Discussions}\label{result}

\subsection{Determinants of adoption: Aggregated}\label{res-agg}

%\begin{landscape}
	\begin{table}[htbp]\centering
\def\sym#1{\ifmmode^{#1}\else\(^{#1}\)\fi}
\caption{Example tables \label{tab-exp}}
\begin{tabular}{l*{4}{c}}
\hline\hline
                    &\multicolumn{1}{c}{(1)}         &\multicolumn{1}{c}{(2)}         &\multicolumn{1}{c}{(3)}         &\multicolumn{1}{c}{(4)}         \\
                    &        reg1         &        reg2         &        reg3         &        reg4         \\
\hline
Headroom (in.)      &    -753.778         &                     &       0.047         &                     \\
                    &   (255.512)         &                     &     (0.691)         &                     \\
Trunk space (cu. ft.)&      58.919         &                     &      -0.184         &                     \\
                    &    (78.718)         &                     &     (0.157)         &                     \\
Weight (lbs.)       &                     &       5.041\sym{*}  &                     &      -0.004\sym{*}  \\
                    &                     &     (0.485)         &                     &     (0.002)         \\
Length (in.)        &                     &     -72.947         &                     &      -0.074         \\
                    &                     &    (49.735)         &                     &     (0.058)         \\
Turn circle (ft.)   &     -39.093\sym{*}  &    -145.993         &      -0.626\sym{***}&      -0.176         \\
                    &     (5.727)         &   (119.909)         &     (0.190)         &     (0.194)         \\
Displacement (cu. in.)&      29.537\sym{**} &       8.902         &      -0.005         &       0.008         \\
                    &     (1.276)         &     (5.020)         &     (0.011)         &     (0.011)         \\
Gear ratio          &    -438.598         &      22.532         &       5.201\sym{**} &       2.446         \\
                    &   (278.602)         &   (576.681)         &     (2.043)         &     (1.749)         \\
Car origin          &    3324.469\sym{**} &    3540.934\sym{**} &      -5.094\sym{***}&      -2.776\sym{**} \\
                    &   (146.881)         &   (115.530)         &     (1.649)         &     (1.285)         \\
\hline
Repair FE           &         Yes         &          No         &         Yes         &          No         \\
Clustered Std Err   &         Yes         &         Yes         &          No         &          No         \\
No. of Obs.         &          69         &          74         &          69         &          74         \\
Outcome mean        &    6146.043         &    6165.257         &      21.290         &      21.297         \\
R2                  &       0.524         &       0.573         &       0.694         &       0.686         \\
\hline\hline
\multicolumn{5}{l}{\footnotesize Standard errors in parentheses.}\\
\multicolumn{5}{l}{\footnotesize * p $<$ 0.10, ** p $<$ 0.05, *** p $<$ 0.01.}\\
\end{tabular}
\end{table}

%\end{landscape}

\autoref{tab-exp} shows the results of xxx. 

Robustness tests are reported in \autoref{fig-random}.

\subsection{Determinants of adoption: Individual technologies}\label{res-ind}

\subsection{Differences in barriers}\label{res-diff}

\section{Conclusion}\label{conclu}

\newpage

\bibliographystyle{apa}
\bibliography{mortgage-polution}

\newpage

\section*{Appendix}\label{appA}

\setcounter{table}{0}
\renewcommand{\thetable}{A\arabic{table}}
\setcounter{figure}{0}
\renewcommand{\thefigure}{A\arabic{figure}}

\begin{figure}[h]
	\centering
	\includegraphics[width=0.7\textwidth]{fig-random.png}
	\caption{Random pic}
	\label{fig-random}
\end{figure}

%\begin{landscape}
%\input{tab-max-base.tex}
%\end{landscape}


%\newpage
%
%\section*{Appendix B} \label{appB}
%
%\setcounter{table}{0}
%\renewcommand{\thetable}{B\arabic{table}}
%\setcounter{figure}{0}
%\renewcommand{\thefigure}{B\arabic{figure}}


\end{document}
